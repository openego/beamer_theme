\documentclass[aspectratio=169]{beamer}
\usepackage[utf8]{inputenc}
\usepackage{lmodern}
\usepackage[ngerman]{babel}

\usetheme{openbeaegon}

\begin{document}

\author{Name des Autors}
\institute{Institutsname}
\title{Titel der Präsentation}

\maketitle

\begin{frame}{Gliederung}
\tableofcontents
\end{frame}

\section{Erstes Kapitel}
\begin{frame}{Titel der Folie}

      	\begin{itemize}
      		\item erste Ebene
      		\item erste Ebene
      		\begin{itemize}
      			\item zweite Ebene
      			   \begin{itemize}
      					\item dritte Ebene
      				\end{itemize}
      		\end{itemize}
      		
      	\end{itemize}
\end{frame}

\section{Zweites Kapitel}
\begin{frame}{Die nächste Folie mit einem ganz langem Titel der zwei Zeilen benötigt}

	\begin{block}{Titel eines Abschnitts}
	Inhalt
	\end{block}
\end{frame}
\subsection{Unterkapitel}


\begin{frame}{Literaturverzeichnis}
\begin{thebibliography}{11}

\bibitem{NEP2019} Netzentwicklungsplan Strom 2030
	\newblock \emph{Version 2019, zweiter Entwurf der Übertragungsnetzbetreiber}
\bibitem{NEP2021} Szenariorahmen zum Netzentwicklungsplan Strom 2035
	\newblock \emph{Version 2021, Entwurf der Übertragungsnetzbetreiber}

\end{thebibliography}
\end{frame}


\end{document}